\begin{figure}[h]
    \centering
    \begin{tikzpicture}
        \node[draw, rounded corners=2mm, inner sep = 0.2cm, fill=orange!0] { 
        \begin{tikzpicture} 
            \node[inner sep = 1mm] (title) { \Large{\bfseries Segment }}; 
            % \draw (title.south west) -- (title.south east);
            \node[at=(title.south), anchor=north, inner sep=3mm, align=left, fill=green!30!black!10, yshift=-0.2cm] (attributes) {
                \begin{minipage}{60mm}
                    \textbf{Attributes}\\
                    \small{0}: \verb|start|\\
\small{1}: \verb|end|
                \end{minipage} 
                };
            % \draw (attributes.north west) -- (attributes.north east);
            \node[at=(attributes.south), anchor=north, inner sep=3mm, align=left, fill=blue!10, yshift=-0.2cm] (methods) {
                \begin{minipage}{60mm}
                    \textbf{Methods}\\
                    \small{0}: \verb|__eq__|\\
\small{1}: \verb|__init__|\\
\small{2}: \verb|__lt__|\\
\small{3}: \verb|__repr__|\\
\small{4}: \verb|current_y|\\
\small{5}: \verb|segment_to_pgf|
                \end{minipage} 
                };
            % \draw (methods.north west) -- (methods.north east);
        \end{tikzpicture}
        }; 
    \end{tikzpicture}
    \caption{Class description - Segment }
    \label{class:Segment}
\end{figure}